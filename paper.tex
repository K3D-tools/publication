\documentclass[preprint,12pt, a4paper]{elsarticle}
\usepackage{amssymb}
\usepackage{lineno}

\usepackage[T1]{fontenc}
\usepackage[utf8]{inputenc}
\usepackage{url}

\journal{SoftwareX}

\begin{document}

\begin{frontmatter}

\title{K3D-jupyter: interactive visualization of 3d data in Jupyter notebook}


\author[cube]{Artur Trzęsiok}
\author[fp]{Tomasz Gandor}
\author[sil]{Marcin Kostur}

\address[cube]{Cubeproject}
\address[sil]{Institute of Physics, University of Silesia, 41-500 Chorzów, Poland}
\address[fp]{Future Processing, 44-100 Gliwice, Poland}
\begin{abstract}
%% Text of abstract 
Ca. 100 words

\end{abstract}

\begin{keyword}
%% keywords here, in the form: keyword \sep keyword
keyword 1 \sep keyword 2 \sep keyword 3

%% PACS codes here, in the form: \PACS code \sep code

%% MSC codes here, in the form: \MSC code \sep code
%% or \MSC[2008] code \sep code (2000 is the default)

\end{keyword}

\end{frontmatter}

\linenumbers

%% main text

Description of your software in maximum 6 pages.

\section{Motivation and significance}
\label{}

Jupyter is a document format and an interactive web tool known as a
computational notebook, which researchers can use to combine software
code, explanatory text, and multimedia resources in a single document.
It has recently become an increasingly popular tool in data
science\cite{perkel_why_2018}, the number of publicly available
Jupyter notebook files on GitHub reach 10 million.

K3D-jupyter is an open-source, MIT-licensed software for scientific
data visualization within a Jupyter notebook. It was built upon
Jupyter widgets for full exploitation of capabilities of the notebook
rich-output architecture.  The main motivation for the development of
this software was the need for an interactive and flexible tool for 3d
visualization of scientific data in the notebook system. The key idea was
to allow the user to visualize data in 3d which is efficient both in
the case of local data exploration as well as for when working
remotely on HPC resources. K3D-jupyter provides basic building blocks,
which serve the purpose of constructing the most of typical
scenarios of data visualization, but at the same time are easy to
use. The main development strategy is to operate on rather simple
structures, and at the same time to provide a straightforward way for
interconnetions to powerfull external libraries.



% Introduce the scientific background and the motivation for developing the software.
% Explain why the software is important, and describe the exact (scientific) problem(s) it solves.
  % Indicate in what way the software has contributed (or how it will contribute in the future) to the process of scientific discovery; if available, this is to be supported by citing a research paper using the software.
% Provide a description of the experimental setting (how does the user use the software?).
% Introduce related work in literature (cite or list algorithms used, other software etc.).


\section{Software description}
\label{}

% Describe the software in as much as is necessary to establish a
% vocabulary needed to explain its impact.


K3D-jupyter is an ipywidget, which is an eventful python object that
has a representation in the browser. It constists of 3d interactive
scene which can contain many different types of objects, like lines,
meshes, volumes, vectors. The visualization is done by the javascript
front-end which uses capabilities of modern WebGL implementation and
allows for smooth interaction. The data displayed by the front-end is
connected to the backend Python process by ipywidgets communication.
Attributes of all obects on the scene, such as point positions, scalar
fields, volumetric data, etc., can be updated from the back-end and 
changes are automatically propagated to the front-end. The Jupyter
notebook is a web application, and the communitation occurs over
websockets. This perhaps is the most important feature of K3D-jupyter:
the same code and software works regardless if it is installed locally
or used remotely. In the contrast to a classical remote-desktop
applications, where there is always significant lag, especially in
video intensive interactions, the browser based K3D-jupyter provides
a constant good experience across many configurations.

There are other projects which exploit Jupyter rich-output
capabilities for data visualisation. Pythreejs\cite{pythreejs} is an
interface to threejs library. Nglview
\cite{10.1093/bioinformatics/btx789}, is a Jupyter widget to
interactively view molecular structures and
trajectories. Ipyvolume\cite{ipyvolume} is a 3d visualization widget
with similar set of features.

Other similar projects: \cite{10.1145/3093338.3104159}






\subsection{Software Architecture}
\label{}

%Give a short overview of the overall software architecture; provide a pictorial component overview or similar (if possible). If necessary provide implementation details.

\subsection{Software Functionalities}
\label{}

Major functionalities of the software.

\begin{itemize}
\item create and display a 3d plot
\item features of display/ snapshots etc.
\item vis-methods - line,mesh,volume etc. - descrie
\item interact with Python attributes to change data
\item present strategies for large data on remote/local situation
\item time animations (2 stategies)
\item interactions (callbacks)
\item specialized features (sparse voxels)

\end{itemize}

\section{Illustrative Examples}
% Sample code snippets analysis (optional) - in both sections 
\label{}

Hightlight some representative features:
\begin{itemize}
\item points (high number)
\item mesh 
\item  volume ct
\item  ?? 
\item vtkcutter example?
\end{itemize}

% Provide at least one illustrative example to demonstrate the major functions.

\section{Impact}
\label{}

\textbf{This is the main section of the article and the reviewers weight the description here appropriately}

%Indicate in what way new research questions can be pursued as a result of the software (if any).
%Indicate in what way, and to what extent, the pursuit of existing research questions is improved (if so).
%Indicate in what way the software has changed the daily practice of its users (if so).
%Indicate how widespread the use of the software is within and outside the intended user group.
%Indicate in what way the software is used in commercial settings and/or how it led to the creation of spin-off companies (if so).

\section{Conclusions}
\label{}

Set out the conclusion of this original software publication.

\section*{Acknowledgements}
\label{}
We acknowledge financial support from the OpenDreamKit Horizon 2020 European Research Infrastructures project (\#676541).


\bibliographystyle{elsarticle-num} 
\bibliography{3d_jupyter.bib}

\section*{Required Metadata}
\label{}

\section*{Current code version}
\label{}

Ancillary data table required for subversion of the codebase. Kindly replace examples in right column with the correct information about your current code, and leave the left column as it is.

\begin{table}[!h]
\begin{tabular}{|l|p{6.5cm}|p{6.5cm}|}
\hline
\textbf{Nr.} & \textbf{Code metadata description} & \textbf{Please fill in this column} \\
\hline
C1 & Current code version & 2.9.1 \\
\hline
C2 & Permanent link to code/repository used for this code version & \url{https://github.com/K3D-tools/K3D-jupyter} \\
\hline
C3 & Legal Code License   &The MIT License  \\
\hline
C4 & Code versioning system used & git \\
\hline
C5 & Software code languages, tools, and services used &  python, javascript \\
\hline
C6 & Compilation requirements, operating environments &  python>=3.6, notebook>5.2
   jupyterlab>1.1,
   ipywidgets,
   pandas,
   scipy,
   numba,
   scikit-image,
   vtk,nibabel \\
\hline
C7 & If available Link to developer documentation/manual & \url{https://k3d-jupyter.org/} \\
\hline
C8 & Support email for questions & marcin.kostur@us.edu.pl  \\
\hline
\end{tabular}
\caption{Code metadata (mandatory)}
\label{} 
\end{table}


\end{document}
\endinput
%%
%% End of file `SoftwareX_article_template.tex'.
