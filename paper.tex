\documentclass[preprint,12pt, a4paper]{elsarticle}
\usepackage{amssymb}
\usepackage{lineno}

\usepackage[T1]{fontenc}
\usepackage[utf8]{inputenc}
\usepackage{url}

\journal{SoftwareX}

\begin{document}

\begin{frontmatter}

\title{K3D-jupyter: interactive visualization of 3d data in Jupyter notebook}


\author[cube]{Artur Trzęsiok}
\author[fp]{Tomasz Gandor}
\author[sil]{Marcin Kostur}

\address[cube]{Cubeproject}
\address[sil]{Institute of Physics, University of Silesia, 41-500 Chorzów, Poland}
\address[fp]{Future Processing, 44-100 Gliwice, Poland}
\begin{abstract}
%% Text of abstract 
Ca. 100 words

\end{abstract}

\begin{keyword}
%% keywords here, in the form: keyword \sep keyword
keyword 1 \sep keyword 2 \sep keyword 3

%% PACS codes here, in the form: \PACS code \sep code

%% MSC codes here, in the form: \MSC code \sep code
%% or \MSC[2008] code \sep code (2000 is the default)

\end{keyword}

\end{frontmatter}

\linenumbers

%% main text

Description of your software in maximum 6 pages.

\section{Motivation and significance}
\label{}

Jupyter notebook is a tool and a document format for reproducible
scientific workflows based on computations. Those notebooks can be
opened with the Jupyter web application which provides rich-output in
the form of browser-side widgets. Those widgets can be make use of all
capabilities of modern browsers, for example WebGL accelerated 3d
graphics. Jupyter architecture ensures sychronization of such
front-end widget with the underlying backend process performing
computations. It has become a very attractive way of accessing
remote HPC resources without sacrificing interactivity. 

K3D-jupyter is an open source, MIT licensed software for scientific
data visualization. It is an ipywidget, i.e. it is a python object
which has its representation in a browser. The main motivation for
development of this software was the need for an interactive and
flexible tool for running accessing computations on both local
workstation as well as remote HPC resources. The front-end of
K3D-jupyter uses capabilities of modern WebGL implementation and
allows for fluent interaction with of the user with visualized 3d
data. Perhaps the most important feature of Jupyter web application is
that there are no differences for the user whether it is used locally
or remotely. In the contrast to classical remote-dekstop applications,
where there is always significant lag, especially in video intensive
interactions, the browser based K3D-jupyter provides constant good
experience across many configurations.





% Introduce the scientific background and the motivation for developing the software.
% Explain why the software is important, and describe the exact (scientific) problem(s) it solves.
  % Indicate in what way the software has contributed (or how it will contribute in the future) to the process of scientific discovery; if available, this is to be supported by citing a research paper using the software.
% Provide a description of the experimental setting (how does the user use the software?).
% Introduce related work in literature (cite or list algorithms used, other software etc.).


\section{Software description}
\label{}

% Describe the software in as much as is necessary to establish a
% vocabulary needed to explain its impact.


\cite{10.1145/3093338.3104159}
\cite{10.1093/bioinformatics/btx789}


\subsection{Software Architecture}
\label{}

%Give a short overview of the overall software architecture; provide a pictorial component overview or similar (if possible). If necessary provide implementation details.

\subsection{Software Functionalities}
\label{}

Major functionalities of the software.

\begin{itemize}
\item create and display a 3d plot
\item features of display/ snapshots etc.
\item vis-methods - line,mesh,volume etc. - descrie
\item interact with Python attributes to change data
\item present strategies for large data on remote/local situation
\item time animations (2 stategies)
\item interactions (callbacks)
\item specialized features (sparse voxels)

\end{itemize}

\section{Illustrative Examples}
% Sample code snippets analysis (optional) - in both sections 
\label{}

Hightlight some representative features:
\begin{itemize}
\item points (high number)
\item mesh 
\item  volume ct
\item  ?? 
\item vtkcutter example?
\end{itemize}

% Provide at least one illustrative example to demonstrate the major functions.

\section{Impact}
\label{}

\textbf{This is the main section of the article and the reviewers weight the description here appropriately}

%Indicate in what way new research questions can be pursued as a result of the software (if any).
%Indicate in what way, and to what extent, the pursuit of existing research questions is improved (if so).
%Indicate in what way the software has changed the daily practice of its users (if so).
%Indicate how widespread the use of the software is within and outside the intended user group.
%Indicate in what way the software is used in commercial settings and/or how it led to the creation of spin-off companies (if so).

\section{Conclusions}
\label{}

Set out the conclusion of this original software publication.

\section*{Acknowledgements}
\label{}
We acknowledge financial support from the OpenDreamKit Horizon 2020 European Research Infrastructures project (\#676541).


\bibliographystyle{elsarticle-num} 
\bibliography{3d_jupyter.bib}

\section*{Required Metadata}
\label{}

\section*{Current code version}
\label{}

Ancillary data table required for subversion of the codebase. Kindly replace examples in right column with the correct information about your current code, and leave the left column as it is.

\begin{table}[!h]
\begin{tabular}{|l|p{6.5cm}|p{6.5cm}|}
\hline
\textbf{Nr.} & \textbf{Code metadata description} & \textbf{Please fill in this column} \\
\hline
C1 & Current code version & 2.9.1 \\
\hline
C2 & Permanent link to code/repository used for this code version & \url{https://github.com/K3D-tools/K3D-jupyter} \\
\hline
C3 & Legal Code License   &The MIT License  \\
\hline
C4 & Code versioning system used & git \\
\hline
C5 & Software code languages, tools, and services used &  python, javascript \\
\hline
C6 & Compilation requirements, operating environments &  python>=3.6, notebook>5.2
   jupyterlab>1.1,
   ipywidgets,
   pandas,
   scipy,
   numba,
   scikit-image,
   vtk,nibabel \\
\hline
C7 & If available Link to developer documentation/manual & \url{https://k3d-jupyter.org/} \\
\hline
C8 & Support email for questions & marcin.kostur@us.edu.pl  \\
\hline
\end{tabular}
\caption{Code metadata (mandatory)}
\label{} 
\end{table}


\end{document}
\endinput
%%
%% End of file `SoftwareX_article_template.tex'.
